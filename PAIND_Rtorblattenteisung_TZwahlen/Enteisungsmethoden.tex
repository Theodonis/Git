\Section{Konzepte zur Eisverhinderung}
\subsection{Verhinderung durch Oberfl�chenstruktur}
Ein Ansatz ist, den Rotor mit einer Oberfl�chenstruktur zu fertigen, welche kein Ansetzen von Eis zul�sst. Bis heute gibt es jedoch keine Oberfl�che, die diese Anforderung erf�llt. Die Vereisung l�sst sich durch Oberfl�chenstrukturen vermindern, jedoch nicht ganz ausschliessen. Weiter haben die Strukturen kurze Lebensdauern und sind schwer zu �berpr�fen.
[Quelle]
\subsection{Mechanische Eisabl�sung}
Eine Methode, die bei kleineren Flugzeugen im Einsatz ist, l�st das Eis durch mechanische Verformung. Luftdruck bl�ht an der Fl�gelvorderkante angebrachte Luftkissen auf und sprengt so das Eis weg. Die Methode beinh�lt zwei wesentlicher Schwierigkeiten. Zum einen m�sste die Verformung genau im richtigen Moment erfolgen. Es muss sich bereits Eis gebildet haben, dieses darf aber die Flugf�higkeit der Drohne nur minimal beeinflussen. Weiter sind die n�tigen Konstruktionen f�r eine solche Vorrichtung an einem drehenden Rotor sehr aufw�ndig.
\subsection{Elektrische Beheizung}
Die in der Aviatik etablierten Systeme basieren auf elektrischem Beheizen. Dabei gibt es gibt zwei Varianten der Ansteuerung. Die eine h�lt nach der Einschaltung den Rotor durch st�ndige Beheizung Eisfrei. Das bedeutet aber einen ununterbrochenen Energie bedarf. In dieser Arbeit sprechen wir von AntiIcing. Weit effizienter ist DeIcing. Die Beheizung erfolgt nur in kurzen Perioden und schmilzt so nur die Verbindung zwischen Eis und Rotor. So l�st sich das Eis ab. Ein Nachteil dabei ist, dass sich das Eis welches sich zwischen zwei Perioden bildet, bereits Einfl�sse auf das Flugverhalten der Drohne haben kann. Weiter ist zu beachten, dass die abgel�sten Eisst�cke zu Sch�den f�hren k�nnen. Beim Superpuma, ein Hubschrauber welche die Schweizer Armee einsetzt, ist der Hauptrotor mit einem DeIcing-System ausgestattet. Der Heckrotor, verf�gt jedoch �ber ein AntiIcing System, da gel�ste Eisst�cke den Hauptrotor oder den Hubschrauber selbst besch�digen k�nnten.
