\section{Versuch zur Ermittlung der thermischen Energie}
Eine erste Versuchsrunde soll die n�tige Energie zum l�sen von Eis zeigen. Mit einem Heizdraht versehene Proben entsprechen dem Aufbau eines Rotorblatt. Auf gefrorene Stoffst�cke sollen die Eisschicht imitieren. Gewicht das an den Stoffst�cke h�ngt, simuliert die Fliehkraft, die am Rotor auf das Eis wirkt. Die Zeit bis sich der Stoff von der Probe l�st soll zeigen welche Leistung pro Quadratzentimeter sinnvoll ist. Leider geben die Resultate keine sinnvolle Resultate. Die Leistung kann nicht in den Zusammenhang zur Leistung gebracht werden. Auch eine Wiederholung des Versuchs in der Klimakammer bei -10�C Lufttemperatur kann die Resultate nicht verbessern. Die

\section{Versuch zur Eisbildung}
\subsection{Versuchsaufbau}
In der Klimakammer soll durch bespr�hen mit Wasser Eis am drehenden Rotor ansetzen. Bei -10�C und einer relativen Luftfeuchtigkeit im Bereich von rund 80 \% dreht der Rotor mit rund 5000 \(\frac{1}{min}\). Eine D�se zerst�ubt Wasser zu Nebel. Eine Datenlogger zeichnet Die Luftfeucht und die Lufttemperatur nahe dem Rotor auf. Ein Kamera zeichnet das Rotorblatt auf. Zus�tzliche Eindr�cke gibt eine W�rmebildkamera.
\subsection{Resultat}
Auf den Bildern der optischen Kamera l�sst sich lediglich das Einsetzen des Nebels erkennen. Der Rotor dreht zu schnell und ist auf den Bildern verschwommen. Auf den Bildern der W�rmebildkamera ist mangels Kontrast auch nicht zu erkennen. Eine HighSpeed Kamera w�rde die Auswertung zuk�nftiger Versuche beg�nstigen.
Die Eisbildung erfolgt unmittelbar nach dem einschalten des Wasser. Die Stromversorgung des Rotorantrieb l�uft augenblicklich in die Strombegrenzung und die Drehzahl f�llt auf rund 4000\(\frac{1}{min}\).
Das Eis setzt fast ausschliesslich and der Vorderkant in Drehrichtung an. Auf der Vorderseite (in Richtung D�se) ist die Eisschicht bis ca. 12mm von der Kante nach hinten. Auf der Hinterseite (von der D�se abgewandt) ragt die Eisschicht bis 26mm von der Kante weg.