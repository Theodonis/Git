\section{statischer Versuch zur Ermittlung der thermischen Energie}
Eine erste Versuchsrunde soll die n�tige Energie zum l�sen von Eis zeigen. Mit einem Heizdraht versehene Proben entsprechen dem Aufbau eines Rotorblatt. Auf gefrorene Stoffst�cke sollen die Eisschicht imitieren. Gewicht das an den Stoffst�cke h�ngt, simuliert die Fliehkraft, die am Rotor auf das Eis wirkt. Die Zeit bis sich der Stoff von der Probe l�st soll zeigen welche Leistung pro Quadratzentimeter sinnvoll ist. Leider geben die Resultate keine sinnvolle Resultate. Die Leistung kann nicht in den Zusammenhang zur Leistung gebracht werden. Auch eine Wiederholung des Versuchs in der Klimakammer bei -10�C Lufttemperatur kann die Resultate nicht verbessern. Die

\section{Versuch zur Eisbildung}
In der Klimakammer soll durch bespr�hen mit Wasser Eis am drehenden Rotor ansetzen. Bei -10�C und einer Luftfeuchte im Bereich von rund 80 \% dreht der Rotor mit rund 5000 \( 1/min\).