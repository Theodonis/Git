\section{Aufbau der Komponenten}\label{Aufbau der Komponenten}
Die Versuchsaufbauten zur Rotorblattenteisung bestehen im wesentlichen aus dem Heckrotorpr�fstand und dem Prototypen. Der Pr�fstand war bereits vorhanden und ist hier f�r zuk�nftig leichtere und schnellere Inbetriebnahme dokumentiert.
\subsection{Heckrotorpr�fstand}
Der Pr�fstand besteht aus einer Halterung, einem b�rsten losen DC-Motor aus der Modeflugbranche und einem Regler zur Ansteuerung. Der Regler ben�tigt eine Speisung von 25V DC und vertr�gt einen Strom von maximal 100A. Sowohl der Regler als auch der Motor sind mit einem L�fter best�ckt. Um die Drehzahl zu detektieren ist ein induktiver N�herungsschalter verbaut und mit einem Tensy 3.2 Entwicklungsboard ausgewertet. �ber eine serielle Schnittstelle gibt das Tensy die Drehzahl auf ein Terminal aus. Ein Servo-Tester stellt die Geschwindigkeit am Regler ein. Der Anstellwinkel der Rotorbl�tter l�sst sich mechanisch an der Spindel verschieben. W�hrend dem Betrieb ist er starr. 

\begin{figure}[H]
\centering
\def\svgwidth{1\textwidth}
\input{Heckrotorpruefstand.pdf_tex}
\caption{�bersicht und Verdrahtung der elektrischen Komponenten des Heckrotorpr�fstand}
\end{figure}

Der induktive N�herungsschalter ist so montiert, dass er bei jeder Umdrehung einmal ein und einmal aus ist. Das Teensy Board Misst jeweils die Zeit zwischen zwei Flanken. Aus jeweils 5 Werten bildet es das arithmetische Mittel. Mit 2Hz druckt das Board das Resultat auf die Console aus.

\subsection{Prototyp}
Der Prototyp �bernimmt Funktion zum ausprobieren, zum messen und zum aufzeichnen. Kernst�ck ist das Tiny K22 Entwicklungsbord. Es ist eine Eigenentwicklung der Hochschule Luzern und verf�gt �ber einen NXP K22FN512 ARM Cortex-M4F Mikrocontroller. Mit einem onboard debugger, dem best�ckbaren SD-Kartenhalter und der kleinen Baugr�sse, deckt es alle Anspr�che\cite{tinyweb}.\newline

\begin{figure}[H]
\centering
\def\svgwidth{0.5\textwidth}
\input{Bilder/Komponentendiagramm.eps_tex}
\caption{�bersicht der Komponenten des Prototyp}
\end{figure}

\subsubsection{Software}
Mit einem FreeRTOS Echtzeitbetriebssystem laufen 4 Funktionen auf der Controller. Das Programm f�hrt die Aktionen in insgesamt 8 Task aus. Die Grundlage f�r das Projekt bieten die Beispielprojekte tinyK22Demo\cite{tinyDemoweb} und tinyK20DataLogger \cite{tinyLoggerweb} von Erich Styger.
Das Bild zeigt die Prozesse, die das Betriebssystem ausf�hrt. Die Schl�ssel symbolisieren mit Mutexen gesch�tzte Daten f�r prozess�bergreifende Zugriffe.

Die Hauptaufgabe ist im Application-Task realisiert und dient zur Ansteuerung eines elektrischen Heizelement. Dabei lassen sich die Heizzeit, die inaktive Zeit und das PWM-Duty Cicycle der Heizansteuerung zur Laufzeit verstellen. Zus�tzlich erm�glicht der Task die EIn- und Ausschaltung der Heizung zur Laufzeit.
\begin{figure}[H]
\centering
\def\svgwidth{1\textwidth}
\input{Bilder/Heating_PWM.eps_tex}
\caption{Logic Analyser Aufnahme vom Ausgang zur Heizungsansteuerung}
\end{figure}

 

