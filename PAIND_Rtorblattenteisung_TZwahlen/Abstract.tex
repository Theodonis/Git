\addsec{Abstract}
%\addcontentsline{toc}{section}{Abstract}
\begin{otherlanguage}{english}
In future the drones from Aeroscout should be capable to fly in high mountain regions. In these conditions it's possible that the rotors freeze which would lead to flight inability. In this project the basis to build a defrosting system should be evaluated. By building a prototype the functions of the defrosting system should be demonstrated.\\
To reach this, information from manned aviation were collected. Tests with a recreated rotor model with integrated heating wires have given an approximation of how much energy is needed to defrost the rotor. In these test the detaching time of frozen textile on the recreated rotor model using different heating powers was measured. Additionally, in a climate chamber at the ETH Zurich it was shown that ice can easily build on the rotor blade while he is turning. Because of missing time the construction of the heatable rotor blade could not be finished and therefore a functioning defrosting system can not be presented. However the prototype without the heatable rotor blades is ready to be tested. He is able to measure air temperature, relative air humidity and save them to an SD-card. Further he's prepared to test a heatable rotor blade with different  heating powers and on-/off times.
\end{otherlanguage}

\newpage
\addsec{Kurzfassung}
Die Drohnen der Aeroscout sollen zuk�nftig f�r Fl�ge im Hochgebirge einsatzf�hig sein. Dabei sind Vereisungen des Rotors m�glich. Die Eisablagerungen f�hren zur Flugunf�higkeit der Drohne. Die vorliegende Industriearbeit soll die Grundlage f�r den Bau einer Enteisungsvorrichtung bieten und mit einem Prototyp diese Funktion demonstrieren.\\
Dazu wurden Erfahrungen aus der bemannten Fliegerei zusammengetragen. Mit einem nachgebauten Rotorblattaufbau mit integriertem Heizdraht wurden Absch�tzungen zur ben�tigten Energie f�r das Eisabl�sen gemacht. Dabei wurden die Abl�sezeiten von aufgefrorenen Stoffst�cken bei verschiedenen Leistungen verglichen. In einer Klimakammer der ETH Z�rich wurde gezeigt, dass sich Eis am Rotor leicht k�nstlich erzeugen l�sst. Eine funktionierende Enteisung konnte mangels Zeit nicht demonstriert werden. Das beheizbare Rotorblatt wurde nicht fertig gestellt, der Rest des Prototyps ist jedoch bereit f�r Tests. Er erm�glicht die Ansteuerung eines beheizbaren Rotorblatts mit verschiedenen Leistungen und verschiedenen Ein- /Auszeiten. Weiter speichert er die Messwerte der Lufttemperatur und der relativen Luftfeuchtigkeit auf einer SD-Karte.