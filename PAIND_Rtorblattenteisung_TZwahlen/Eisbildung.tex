\section{Eisbildung}
Die Problematik von Eisbildung an Rotorbl�ttern entsteht zum gr�ssten Teil durch unterk�hlte Wasser Tr�pfchen in der Luft. Sie haben eine Temperatur von weniger als 0�C, sind aber trotzdem im Fl�ssigen zustand. Diese Tropfen gefrieren wenn sie in Kontakt mit bereits bestehenden Kristallstrukturen kommen oder bei Ersch�tterungen. Sie gefrieren also augenblicklich beim Aufprall auf den Rotor und haften daran, wenn des Oberfl�chentemperatur unter 0�C liegt.
Es k�nnen sich verschieden Arten von Eis bilden. Es gibt Raueis, Klareis und Mischeis, die Mischung der beiden anderen. Die Eisarten wirken sich aber in diesem Zugsamenhang lediglich darauf aus, wie leicht sich das Eis l�sen l�sst. Sie werden nicht weiter von einander unterschieden. 
